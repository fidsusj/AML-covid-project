\section{Conclusion} \label{conclusion}

Coming to a conclusion, the main research question was whether a machine learning model can be trained to predict the next possible \ac{SARS-CoV-2} mutations. 

For answering the research question a new dataset based on raw data from \ac{GISAID} was created. First the raw data is selected and downloaded as described in chapter \ref{fundamentalsC} and chapter \ref{approachAa}. Next a dataset consisting of parent-child genome pairs is generated through a reusable pipeline for creating evolutionary datasets as described in the chapters \ref{fundamentalsD}, \ref{approachAb} and \ref{approachAc}. Data insights of the generated dataset are investigated in chapter \ref{experimentsA}.

Based on related works introduced in chapter \ref{fundamentals}, a model ar\-chi\-tec\-tu\-re consisting of a transfomer-based \ac{GAN} framework was chosen (chapter \ref{approachC}). This ar\-chi\-tec\-tu\-re was influenced by Berman et al. \cite{Berman2020} and builds upon its improvement proposal to incorporate transformers instead of \acp{LSTM} as a central novelty in this paper leveraging the attention mechanism and parallel processing power of the transformer architecture to achieve higher quality predictions. 

The selected ans implemented model is trained on the generated dataset for \ac{SARS-CoV-2} as de\-scri\-bed in chapter \ref{approachD}.

For answering the research question the trained model is evaluated. The short answer is, that the trained model is definitely able to generate possible next genome sequences demonstrating a \ac{BLEU} score of 87.42\% and a sequence true positive rate of 31\%. Evaluation details and comparisons to existing works can be found in chapter \ref{experimentsB}.


\vspace{0.5cm}

As an outlook, there are some improvements and further investigations we would like to evaluate in future work. As realized while generating data insights of the dataset, lots of our data instances are very similar to each other. To really model the worldwide development of \ac{SARS-CoV-2} mutation events, the raw dataset should be based on a subsampling of all available global data instances. Further improvements could be achieved by using a computing cluster for generating a larger dataset and for training the model on the full \ac{SARS-CoV-2} genome and not on a subset of the genome. Moreover, beam search instead of greedy decoding during the training phase could be further examined.

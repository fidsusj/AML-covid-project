\section{Approach} 
\label{approach}

TODO: Pipeline image

\subsection{Dataset creation} 
\label{ch:approachA}

hier irgendwo das Zielschema des Datensatzes beschreiben

\subsubsection{Raw data selection from GISAID}
\label{ch:approachAa}

focus: Germany from 4.5. - 6.8. (new variant arised in the recent past -> lambda, delta, ...)
only Germany, to make it possible to handle the data
about 35000 genomes in our raw dataset

beispiel record: genome sequence and metadata


\subsubsection{Generation of a phylogenetic tree}
\label{ch:approachAb}


\subsubsection{Phylogenetic tree to dataset}
\label{ch:approachAc}


\subsection{Data Preprocessing} 
\label{ch:approachB}

two steps:
- to make the dimensionality managable not the whole 30000 nucleotides are evaluated. We take a subpart of X nucleotides from position A to B
- Transform string to numeric for model input

\subsubsection{Dimensionality reduction by selecting subpart of the genome}
\label{ch:approachBa}


\subsubsection{Transform genome sequence to numeric model input}
\label{ch:approachBb}



\subsection{Model architecture} 
\label{ch:approachC}

\subsection{Training process} 
\label{ch:approachD}


\newpage

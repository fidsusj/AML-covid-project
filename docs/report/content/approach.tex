\section{Approach}  \label{approach}

\subsection{Dataset Creation}  \label{ch:approachA}
% Choosing latest mutations that are the most spread
% created three types of sequence pairs: (real, real), (real, generated), (real, unreal)

\subsection{Data Preprocessing}  \label{ch:approachB}
% TODO: Pipeline image
% Word size 3 is just a hyperparameter (see discussion of the corresponding paper)!  But maybe biologically valid because auf amino acids.
% DNA2Vec?
% Phylogenetic tree with "Fasttree"
% Biopython packages?
% Levenshtein distance

\begin{itemize}
	\item DNA Sequencing
	\item DNA Sequence Tokenization for Amino Acid Dictionary
	\item (DNA Sequence Padding not necessary as model can handle input of arbitrary length)
	\item Phologenetic Tree and final dataset metrics
\end{itemize}

\subsection{Model Architecture}  \label{ch:approachC}
% Use Keras for implementation + Tensorflow 2
% https://github.com/tensorflow/nmt

% https://towardsdatascience.com/neural-machine-translation-using-seq2seq-with-keras-c23540453c74
% Plot accuracy and loss for training and validation
% Teacher forcing, early stopping?
% State size 256, 512, 1024?
% Search other code tutorials
% Telegram messages Nils

% TRANSFORMER Tutorial: http://nlp.seas.harvard.edu/2018/04/03/attention.html !!!
% OR: https://colab.research.google.com/github/tensorflow/tensor2tensor/blob/master/tensor2tensor/notebooks/hello_t2t.ipynb
% use prebuilt transformer


\subsection{Training Process} \label{ch:approachD}

% Loss functions and teacher forcing
% Categorical cross-entropy loss for seq2seq, Wasserstein loss for GAN
% Identify changes between sequences using the diff-match package from Google
% See MutaGAN 6.2 Model training
% Replay buffer
% Dealing with the mode collapse problem

\newpage

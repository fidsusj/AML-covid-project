\section{Approach}  \label{approach}

\subsection{Dataset Creation}  \label{ch:approachA}
% Choosing latest mutations that are the most spread

\subsection{Data Preprocessing}  \label{ch:approachB}
% TODO: Pipeline image
% Word size 3 is just a hyperparameter (see discussion of the corresponding paper)!  But maybe biologically valid because auf amino acids.
% DNA2Vec?

\begin{itemize}
	\item DNA Sequencing
	\item DNA Sequence Tokenization for Amino Acid Dictionary
	\item DNA Sequence Padding
\end{itemize}

\subsection{Model Architecture}  \label{ch:approachC}
% Use Keras for implementation + Tensorflow 2
% https://github.com/tensorflow/nmt

% https://towardsdatascience.com/neural-machine-translation-using-seq2seq-with-keras-c23540453c74
% Plot accuracy and loss for training and validation
% Teacher forcing, early stopping?
% State size 256, 512, 1024?
% Search other code tutorials
% Telegram messages Nils


\subsection{Training Process} \label{ch:approachD}

% Loss functions and teacher forcing
% Categorical cross-entropy loss

\newpage

\section{Proposal 1: Machine Learning based prediction of the next SARS-CoV-2 variants and simulation of vaccine effectiveness}

%1. Which scientific questions do you want to answer, and why are they interesting?
\textbf{1. Idea and research question}

There is hope for an end of the pandemic due to the vaccines. All vaccines are developed against the wild type of the virus. Through the mutation processes the danger, that a variant occurs, against which the vaccines  are no longer effective, arises. This process is known as viral escape. This proposal aims to detect this development faster to enable a faster and better response to new variants.

So there are two scientific questions (which can also be separated):
\begin{enumerate}
	\item Can the next possible SARS-CoV-2 variants be predicted by a Machine Learning model?
	\item Based on the  predicted possible next variants can the effectiveness of the vaccines be simulated?
\end{enumerate}

%2. Which relevant papers did you find (either in the spreadsheet or by your own search)? What are their pros and cons? Which among these papers do you propose to build upon, and why?
\textbf{2. Related work}

Already some works exist in the area of predicting virus mutations using Machine Learning techniques. Hie et al. \cite{Hie284} applied methods developed for \ac{NLP}.  According to their work escape mutations look different to the immune system, but have the same viral infectivity. The analogy from the \ac{NLP} area are word changes, which change the meaning of a sentence, but the grammaticality remains. Through their work, they have managed to generate a connection between natural language and viral evolution.

Even before the SARS-CoV-2 pandemic, research was done in this area, e.g. from Salama et al. \cite{salamaPredictionVirusMutation2016}. They used neural networks for predicting new mutations and rough set techniques for detecting patterns in mutations. Furthermore they validated their approach for the Newcastle virus and achieved an accuracy of 75\%.

\ac{GAN} achieve great results in image generation. Berman et al. applied a \ac{GAN} in \cite{berman2020mutagan} to generate \ac{DNA} sequences.


%3. What methods will you try, and why do you consider them promising to answer the questions?
\textbf{3. Approach}

After preprocessing the raw SARS-CoV-2 genomes we would like to train a Machine Learning model for predicting the probable next mutations. The Machine Learning model could be a standard neural network (as in  \cite{salamaPredictionVirusMutation2016}), a GAN (as in \cite{berman2020mutagan}) or a neural network language model (as in \cite{Hie284}). Due to a lack of time we haven't decided yet for one Machine Learning model.

%4. What will your data sources be? How large are the data sets and how high do you think the quality is? Can you use simulated data when real data is scarce?
\textbf{4. Data sources}

For predicting the next possible SARS-CoV-2 variants the SARS-CoV-2 genome development from the past can be used as data source (genomic time series data). Data sources are available through the GISAID initiative \cite{gisaideditorGISAIDMission}. In April 2021 over one million SARS-CoV-2 genomes are available via GISAID \cite{maxmenOneMillionCoronavirus2021}. Based on this data the Nextstrain project provides a visualization how the SARS-CoV-2 genome evolves: \href{https://nextstrain.org/ncov/global?label=clade:19A}{Nextstrain}

%In terms of data quality, when working with genomic data one should always have in mind, that genotyping errors are possible.


%5. What computational resources do you need (e.g. GPUs)? How will you get access to these? How much time will the computations need?
\textbf{5. Computational resources}

The needed computational resources and time depends on the concrete chosen Machine Learning approach. As described above due to a lack of time we haven't decided yet for one Machine Learning approach. For further research and the decision for one Machine Learning approach we keep this in mind.


%6. What difficulties do you anticipate in the project?
\textbf{6. Probable difficulties}

Challenges could arise due to the amount of data (one SARS-CoV-2 genome consists of about 30.000 bases => TS size = data instances * 30.000). This can be encountered with less data instances or a compression of SARS-CoV-2 genome e.g. through DNA2vec \cite{ng2017dna2vec} (represents pieces of \ac{DNA} as vector).

\newpage
